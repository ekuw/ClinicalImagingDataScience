\documentclass{article}

\usepackage[margin=1.0in]{geometry}
\usepackage{authblk}
\usepackage{url}
\usepackage[sorting=none]{biblatex}
\addbibresource{data-hazards.bib}


\title{
  Elizabeth Blackwell Institute Project Report:\\
  Stroke Imaging And Clinical Database For Artificial
  Intelligence
}
\author[1]{Emma Si\^{a}n Kuwertz}
\author[2,3]{Philip Clatworthy}
\affil[1]{Jean Golding Institute, University of Bristol}
\affil[2]{North Bristol NHS Trust}
\affil[3]{Bristol Medical School, University of Bristol}
\date{November 2021}

\begin{document}
\maketitle
\newpage
\tableofcontents
\newpage
\section{Introduction}

Stroke is a major cause of disability in adults. Treatment
decisions rely on the ability of clinicians to combine insights
gleaned from medical imaging (e.g. computerized tomography (CT)) with
relevant clinical information and patient medical histories to anticipate the outcome of different treatment
scenarios. Decision support tools exist, but these are typically able
to consider information from isolated sources and do not allow the
integration of medical images. The \textit{Stroke Imaging and Clinical
Database for AI} project aims to extract provide a linked dataset,
using clinical and medical imaging inputs from North Bristol NHS Trust (NBT). 
This will facilitate the development of multi-modal models to explore the potential for stroke treatment outcome
prediction.\\



\section{Project achievements}

This project was granted funding for 6 months by the Elizabeth
Blackwell Institute (EBI) Health Data Science research strand (with
support from the Wellcome Trust Institutional Strategic Support Fund
(ISSF)), commencing in June 2021.

The intended outcomes of this work are as follows:
\begin{itemize}
\item 	A repository of routinely collected NHS stroke clinical and imaging data, curated to enable the development of artificial intelligence software for clinical applications. 
\item 	Design and plan for a data pipeline, centred around the NBT Data Warehouse, PACS and Electronic Patient Record, scalable to include other health, social care and research data sources. 
\item 	Workshop with clinical and commercial partners to consult on
  repository specifications and identify target clinical research
  projects in stroke, and where possible initiate pilot projects using
  the repository produced in (1) above.
\end{itemize}

\noindent Towards this, during the 6 month period funded by the EBI, the
following aims were pursued:
\begin{enumerate}
\item Draft project protocol to gain NBT sponsorship.
\item Prepare documents for Health Research Authority approval.
\item Produce a specification for the data repository.
\item Conduct a small pilot project using data.
\item Scope opportunities for further funding.
\end{enumerate}

\noindent The progress made towards these aims is summarised in the following
section.
\subsection{Protocol and sponsorship}

The protocol for this project has been drafted and was submitted to
NBT research sponsor for review on 16/11/2021 (R\&I number 5055). This was prepared with
support from NBT.\\

\subsection{Documents for Health Research Authority Approval}

\subsubsection{Risk Register}
A risk register was completed following iteration with the NBT
research sponsorship team. This included confirming the peer review of
the project obtained during the EBI funding panel review. It is
being considered whether the updated protocol document (once the next round of feedback from NBT
sponsor is implemented) will be shared with a group of selected
experts for a formal peer review.\\

A key question during risk assessment and project approval concerns
the sample size of the dataset. We expect approximately 10,000 people
in the cohort initially, which is constrained by data availability
(PACS contains images for the past 15 years, corresponding to
approximately 10,000 patients).
Large samples are generally required to develop machine learning
algorithms, and determining optimal sample size will feature as an
intermediate outcome of this project. While 10,000 will provide a
reasonable statistical basis for initial studies, as R\&D progresses
using the existing data the optimal sample size will likely be
reviewed.\\

\subsubsection{Data Protection Impact Assessment}


During the risk and data protection impact assessment it was important to identify any
sensitive data fields that would need to be accessed during database
construction (even if such fields may be subsequently dropped later in
the data processing pipeline and therefore not stored in the research
database). In most cases NHS number is enough to cross-reference
records from different tables / databases.
Here, only those persons directly involved in the data extraction and
linkage procedure (e.g. clinical staff executing database queries or
those linking the output tables together) will have access to
identifiable information. Any information only needed for the linking
process will be removed from the linked outputs. Some personal data,
such as date of birth, is necessary for algorithm development (as age
could be an important indicator when predicting patient treatment
outcomes). The sensitive fields of relevance have been identified and
reasoned in the project protocol document. The data protection impact
assessment form for this project was signed by the responsible person
at NBT on
18/10/2021.\\

\subsubsection{IRAS Form}

In parallel to the data protection impact assessment, risk register
and protocol draft, an IRAS form has been completed (IRAS number 302301) ready for
submission to the research sponsor (and subsequently HRA). \\

\subsubsection{Proof of Indemnity Insurance}

Following instructions from the University of Bristol's Research
Governance
team\footnote{\url{http://www.bristol.ac.uk/secretary/insurance/clinical-research-insurance/}}
it was confirmed that the University's Public Liability insurance
applies for this research project: \textit{The University of Bristol's
  Public Liability insurance policy provides an indemnity to our
  employees for their potential liability for harm to participants
  during the conduct of the research.}
This follows, since this study falls under the Minimal Intervention Studies heading in the Data-only studies category. \\

The study has been registered with the Research Governamce team via
the completion of the \textit{Research Registration Checklist}, which
is available through the Research Governance
webpages\footnote{http://www.bristol.ac.uk/red/research-governance/}.
A summary of the information entered into this checklist can be found
in Appendix~\ref{app:research_reg_checklist}.

\subsection{Data specification}

This study will require the bringing together of data from the
following sources:
\begin{itemize}
\item	North Bristol Trust data warehouse: A full data extract for a 15 year period (dictated by data availability), which includes: 
  \begin{itemize}
  \item	Subset of EPR data fields in Lorenzo [including NHS number and hospital number for data linkage purposes] 
  \item	Subset of RIS data fields [Radiology data including exam code (both diagnostic and interventional procedures), procedure dates, imaging reports]
  \item	Subset of operative procedure data fields e.g. neurosurgical
    procedures, including procedure dates
  \item	Diagnosis coding
  \end{itemize}
\item	North Bristol Trust data entered into the Sentinel Stroke National Audit Programme database 
\item	Imaging data from North Bristol Trust PACS, including: CT images, MR images, Digital Subtraction Angiography, Vascular ultrasound (e.g. carotid duplex).
\end{itemize}

During the 6 month scoping period clinical analysts at NBT have
provided information as to the data tables and fields available at the
NBT Data Warehouse. This has enabled a first draft of the data fields
to be queried in order to select the data extract to be studied.\\

In order to further prune the dataset and provide a clear data
specification, access to the data is required by the research
team. For the data science experts (Emma Kuwertz) involved in data
specification and processing, this requires an NHS letter of
access. Relevant paperwork to secure this has been completed, but the
research sponsor will need to approve the protocol and secure HRA
approval prior to the letter of access being issued.\\

\subsubsection{Sensitive data}

During the Data Proteciton Impact Assessment a number of fields that
corresponded to either personal
identifiable data or sensitive (special category) data were flagged as being potentially needed for data
processing purposes. An effort has been made to identify the minimum possible sensitive
data fields needed for data querying and linkage. Once a sensitive
data field has served its purpose it will be dropped from the dataset
at the next step of the processing chain. \\

NHS number is needed in
order to query and link patient records from the PACS system to
clinical records from the NBT data warehouse.
Date of birth and Age are needed for data quality checks on procedure
dates and age at time of procedure, though day of month will be
dropped early in the data processing chain (MMYY should provide
adequate accuracy to verify the age of a patient at the time of a
particular CT scan image). Following this verification date of birth
will be dropped from the extract prior to pseudonymisation.
Gender is of interest for research purposes and will be
retained in the anonymised dataset.  Diagnosis is also of relevance to
predictive modelling research. Including diagnosis in the data will
allow for research into the training of predictive models to extract
features common to images and clinical records with a given diagnosis.


\subsection{Pilot project}

Without HRA approval it has not been possible to carry out a pilot
project using the proposed dataset. There has, however, been
discussions and connections made with some interested academics. This
has resulted in some ideas for pilot projects when the data repository
is ready in a preliminary form. This has resulted in the preparation
of two grant applications to date. \\

Within this consultation the following preliminary research questions
have been identified for IRAS form submission for HRA approval:
\begin{itemize}
\item \textbf{Primary questions:}
  \begin{itemize}
  \item Identify cases with and without stroke, as well as stroke subtypes, including evaluation of different methods and combinations of methods to achieve this.
  \item predict outcomes with and without treatment, including
    modelling the benefit of treatment.
  \end{itemize}
\item \textbf{Secondary research questions:}
  \begin{itemize}
  \item Can NPL be used on imaging reports in order to efficiently automate labelling of images for the training and validation of deep learning models?
  \item Can a secure data extraction, linkage, and anonymisation
    process be automated to provide a data pipeline to feed a
    research database for clinical and image data?
  \end{itemize}
\end{itemize}



\subsection{Further funding opportunities}

During this project scoping phase several opportunities for future
funding have been identified and pursued.
\begin{itemize}
\item A substantial imaging strand has been formed within the NIHR
  Biomedical Research Centre funding bid (invitation for interview in
  March 2022).
\item An expression of interest has been submitted for an MRC funding
  call \textit{AI for better biomedical research} to investigate natural language processing
  applications within this dataset [unsuccessful].
\item An expression of interest has been submitted to Southmead Hospital Charity
  Research Fund to use this dataset to study cerebral venous thrombosis.
\end{itemize}

Further funding sources are being sought to facilitate planned future
work surrounding database design and consultation with academics,
clinicians and the public.

\section{Data Hazards Workshop Participation}

This project includes the use of patient information collected at NBT as a part
of routine care. Public patient involvement (PPI) is an essential
requirement of this research, and will help to ensure that the project
will be designed and implemented in such a way as to be acceptable to
the public. Community engagement also enhances the ethical
review process, facilitating open discussions about the risks and
benefits of the research and potentially flagging issues that can be
addressed early on in the project development.\\

As a part of internal ethics review and public consultation, this
project was presented and discussed at the first \textit{Data Hazards
  Workshop}~\cite{data-hazards-workshop} in September 2021. Here, this
project was presented and discussed within a small group of public
participants consisting of doctoral students, academics and data
professionals. An article
reporting on the discussions and considerations raised at the
workshop was subsequently produced and will form an important part of
the project PPI documentation.\\

The data hazards discussions highlighted concerns regarding the
following data hazards:

\begin{itemize}
\item Reinforcing existing bias
  \begin{itemize}
  \item The dataset is selected based on locality (limited to NBT)
    and procedure (neuroimaging) and so will be demographically
    biased.
  \item The sample may be intrinsically biased due to internal composition.
  \end{itemize}
\item Danger of misuse
  \begin{itemize}
  \item The immediate project output is a dataset and not an
    algorithm, therefore it is difficult to predict all possible
    future uses for this dataset. How can misuse be foreseen and prevented?
  \end{itemize}
\item Lack of informed consent
  \begin{itemize}
  \item Data used here is collected as a part of routine care, making
    the collection of explicit consent from individuals difficult / unrealistic.
  \end{itemize}     
\item Lack of community involvement
  \begin{itemize}
  \item In the absence of explicit patient consent, consultation with
    patient communities is important and should be a priority.
  \end{itemize}     
\end{itemize}


\section{Consultation workshop plans}

Once the project data is in hand and in-principle available on the
University of Bristol's RDSF, a series of consultation workshops are
planned. It is envisaged that this series will be composed of one
preliminary workshop, followed by two hackathon-style events.
The purpose of these will be as follows:

\begin{enumerate}
\item Gain feedback from the academic and clinical
  community on how they envisage using the dataset in their
  research. 
\item Identify additional research questions that will be studied using the data
  resource.
\item Understand how clinical and computer science researchers
  currently interact with data: which tools are predominantly used?
\item Gain insight into how researchers will want to query the
  data.
\item Identify any core data fields which will be most commonly
  in use.
\end{enumerate}

The above points will further enable the design and specification for
a queriable database. Addressing the above will allow planning for
optimal data extraction and filtering from the database, and will
essentially inform database design and construction.\\

The initial preliminary workshop will be an opportunity to bring
together all co-applicants on this project and provide interactions
between clinical researchers, software engineers and computer
scientists. This will be an opportunity to present this preliminary
data offering and scope use cases and collaborative pursuits. The plan
for the preliminary workshop is as follows:\\

\noindent\textbf{\underline{Preliminary workshop}}\\

\textbf{Foreseen attendees:}
\begin{itemize}
\item Dr Marcus Bradley, Consultant interventional neuroradiologist
  and Clinical Lead for Neuroradiology, North Bristol Trust
\item Rhona Galt, Director of Transformation and Innovation/AI lead,
  North Bristol Trust
\item Dr Emma Kuwertz, Data Science Specialist, Jean Golding
  Institute, University of Bristol
\item Dr Phil Clatworthy, Honorary Senior Lecturer and
  Consultant Stroke Neurologist, North Bristol Trust
\item Prof. Kate Robson-Brown, Director, Jean Golding
  Institute, University of Bristol
\item Mr Kumar Abhinav, Consultant Neurosurgeon, North Bristol Trust
\item Prof. Rob Hinchliffe, Professor of Vascular Surgery and
  Consultant Vascular Surgeon, North Bristol Trust
\item Dr Richard Ibitoye, Honorary Consultant Neurologist,
  Gloucestershire Hospitals NHS Foundation Trust
\item Prof Majid Mirmehdi, Professor of Computer Vision, Department of
  Computer Science, University of Bristol
\item Dr Edwin Simpson, Lecturer of Computer Science, Department of
  Computer Science, University of Bristol
\item Dr Conor Houghton, head of Neural Computation group, Faculty
  of Engineering, University of Bristol
\item Dr George Harston, Chief Medical and Innovation Officer,
  Brainomix Limited, Oxford, UK
\item \textit{Invite research software engineering representatives from
    University of Bristol's Advanced Computing Research Centre}
\end{itemize}

\textbf{Agenda}
\begin{itemize}
\item    Present data specification and availability \textit{(Phil
    Clatworthy \& Emma Kuwertz)}
\item    Introduce JGI and current health data science collaboration \textit{(Phil
    Clatworthy, Emma Kuwertz, Kate Robson-Brown)}
\item    Present research questions in stroke \textit{(Phil
    Clatworthy \& others)}, vascular surgery \textit{(Rob Hinchliffe
    \& others)},
  neuroradiology \textit{(Marcus Bradley \& others)}
\item Questions to participants (in advance):
  \begin{itemize}
  \item Participant research ideas
  \item Participant research area questions / problems / issues that
    could be addressed (or studies that could be facilitated) using
    the project data resource.
  \end{itemize}
\end{itemize}

It is expected that discussions during the preliminary workshop will
result in the identification of additional data preparation /
processing steps necessary to facilitate researcher interaction with
the dataset.\\

\noindent\textbf{\underline{Hackathons}}\\

Ahead of the hackathons, the following steps will be
taken and resources
prepared for participants:

\begin{itemize}
\item Data quality checks and cleaning will be performed
\item Data will be linked together appropriately and access to the
  dataset on RDSF will be secured for named researchers
\item Metadata will be prepared to describe the dataset contents and layout
\item Data querying examples/scripts will be provided for hackathon participants
\item Instructions for data access will be provided to hackathon participants
\end{itemize}

During the hackathons, researchers will access the data and begin
preliminary analyses / investigations in their chosen research
area. This hands-on experience with the dataset will help to expose
any additional issues with the dataset and identify any missing or
perhaps superfluous information. In this way, the first hackathon is
expected to further inform dataset design. \\

The second hackathon will occur following a first round of feedback
from researchers regarding their experience of using the dataset. This
means that in the second hackathon access to a new iteration of the
dataset will be provided for researchers. Researchers will also
provide feedback on important questions like sample-size and
composition appropriateness. 

\section{Next steps}


\subsection{Patient Public Involvement}


During the expert consultation at the Data hazards workshop there was speculation as to what degree the general public expect that
their personal and medical information might be used in research
without their explicit permission (even where data is collected as a
part of routine care).
In general there was a notion that the public generally accepted that
their medical data would likely be ued to further the public good,
with the understanding that no identifiable information would be
compromised. Still, it is important to engage with patients in this
space. The Bristol Health Partners Stroke Health Integration Team
(HIT) has a Service User Group involving people across BNSSG with
experience of stroke and stroke services.
They provide PPI input for research projects and will be given the
opportunity to be involved in this project in the future,
particularly in the areas of research design, analysis and
dissemination.

\subsection{Data specification}

Members outside the direct care team are unable to access the data
extracted from NBT until HRA approval is secured. This means that the
EBI-funded phase of this project has been limited to consultation with
NBT analysts regarding data availability and type. Following HRA
approval, data will be extracted from the NBT Data Warehouse and the
cleaning and linkage process can proceed. This will be an exploratory
procedure initially, with quality checks being performed to assure the
reliability of data fields, and studies being carried out to validate
the data presentation (how should the data be \textit{shaped}, what is
the most sensible way to organise and link the data).\\

\subsubsection{Data linkage}

The data will be extracted from different tables within the NBT Data
Warehouse, and will need to be combined / linked. This is envisaged to
take place initially using NHS numbers where possible, but it is
likely that other identifiers will be needed in some cases (e.g. to
link procedure codes with procedure scans and reports). A linkage
method will need to be established and tested using an initial data
extract. This will be done within the NBT network.

\subsubsection{Data quality}

Following the establishment of a linkage procedure, the procedure
itself will need to be validated. Sanity checks will be specified and
carried out on the linked data to ensure that data has been linked as
intended and expected.\\

Futher, individual data fields will need to be quality assured to
determine reliability. For example, if some fields contain a mjority
of missing data, it should be understood why this is. Some fields may
contain contradictory information, which should be investigated. This
will allow the most reliable fields to be identified, and the most
unreliable to be flagged (and possibly removed).

\subsubsection{Automation}

Once a reasnoably final dataset has been constructed, with clearly defined
data fields and quality assurance and linkage procedures
established, one should aim for automation. This should be a
consideration throughout the exploratory and iterative data extraction
process. Challenges to automation should be identified and
considered. In order to scale this resource in the future, a clearly
defined and reproducible recipe for database creation should be
curated. 


\subsection{Privacy and ethics}

There are several areas under ethics, privacy and security that will
be further studied and addressed. These include data investigations
and subsequent metadata to document data composition and highlight
internal dataset bias and arrangements for future data sharing and data access.

\subsubsection{Sample bias}
The discussions from the Data Hazards workshop highlighted the need
for clear documentation of sample composition and available data
fields. An important question to be investigated with the
researcher-access database is \textit{``what metadata should be provided
  alongside the database to facilitate the mitigation of bias when the
  data is used in research?"} At the very least the following
database statistics will be made available:
\begin{itemize}
\item What is the sample size?
\item What is the gender composition?
\item What is the ethnicity composition?
\item What is the hospital-level composition?
\item What is the procedural composition?
\item What is the age composition?
\end{itemize}

\noindent Clinicians will beconsulted to exploit domain knowledge that may highlight clinical
selection bias.


\subsubsection{Misuse}

To guard against potential misuse of this data resource, a clear data
access request and approval procedure will be drafted. It is envisaged
that new access requests will be submitted to the project PI (Phil
Clatworthy). These requests will need to include a clear proposal
describing the intended use of the data in research. If granted, access to the
data will be permitted for purposes related to the proposed research
only, and will be given for the proposed project duration.\\

To minimise the potential for data to be linked with other unintended data
sources out of context, the data fields made available in the dataset
will be strictly controlled. The planned consultation workshops will
provide a pruning exercise, whereby superfluous data fields will be
dropped and motivated data fields kept. \\

\subsubsection{Security}

Further engagement with NBT and University of Bristol IT services will
help to inform and solidify the data sharing arrangements for the
research database. The RDSF will be used at the University of Bristol
during dataset scoping and pilot phases, however a queriable database
would be more efficiently accessed from elsewhere (RDSF is not
optimised for efficient data access). Planning and executing this
phase of database design will be important following initial dataset
preparation (e.g. in parallel with / informed by the series of
consultation workshops).




\appendix
\section{Univesity of Bristol: Research Registration Checklist}
\label{app:research_reg_checklist}
\begin{small}
  \begin{center}
    \begin{tabular}{p{0.3\linewidth}p{0.6\linewidth}}
      \hline
      \multicolumn{2}{c}{GENERAL}\\
      \hline
      
      Submitter &  Emma Kuwertz \\
      CI / PI &  Phil Clatworthy \\
      Title &  OUTCOME PREDICTION AND DECISION SUPPORT FOR STROKE USING ROUTINE NHS DATA\\
      Student? &  No\\
      Start &  2021-06-01\\
      End &  2025-05-31\\
      Finance &  1415172\\
      \hline\hline
      \multicolumn{2}{c}{RISK LEVEL}\\
      \hline
      CTIMP? & No\\
      Device? &  No\\
      Human Tissue? No\\
      Sites &  ["staff, premises or resources of one or NHS Trusts?"]\\
      \hline\hline
      \multicolumn{2}{c}{SPONSORSHIP \& REC}\\
      Sponsorship &  My project does meet one or more of the above criteria - it will be Sponsored by another organisation.\\
      Sponsorship, if other &  North Brisol NHS Trust\\
      NHS REC? &  My project does not meet any of the above criteria BUT does involve NHS staff or facilities. (In this case you will require HRA approval.)\\
      UoB REC? &  I will apply for UoB School/Faculty REC using the online application system:  orems.bristol.ac.uk\\
      S/FREC, if applicable & \\
      S/FREC reference, if applicable & \\
      REC, if Other & \\
      \hline\hline
      \multicolumn{2}{c}{INSURANCE}\\
      \hline
      Non PL categories & \\
      NHS Indemnity? & \\
      MDU? & \\
      CI / PI &  Phil Clatworthy\\
      Employer & \\
      Honorary Contract? & Yes\\
      Other staff & \\
      Other info & \\
      \hline\hline
      \multicolumn{2}{c}{FUNDING \& PEER REVIEW}\\
      \hline
      Funded? &  Yes\\
      Funder &  EBI\\
      Amount &  £10000\\
      Peer Review &  I will be seeking peer review.\\
      Reviewer(s) &  The study team has sufficient expertise to ensure
                    the scientific merit of the project. Please let us
                    know if we have to provide a formal report. The project was also reviewed by EBI, who awarded the initial funds. Data processes have been through the formal information governance processes at NBT. Data Protection Impact Assessment form can be made available on request.\\
      \hline\hline
      \multicolumn{2}{c}{ADOPTION}\\
      \hline
      Portfolio &  No\\
      BTC &  No\\
      \hline\hline
      \multicolumn{2}{c}{CONTACT}\\
      \hline
      CONTACT
      Name &  Emma Kuwertz\\
      Role &  Other Investigator\\
      Department &  Jean Golding Institute\\
      Contact &  emma.kuwertz@bristol.ac.uk\\
      Date &  2021-11-22\\
      \hline
    \end{tabular}
  \end{center}
\end{small}

\printbibliography
\end{document}
