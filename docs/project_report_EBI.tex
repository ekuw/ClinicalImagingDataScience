\documentclass{article}

\usepackage[margin=1.0in]{geometry}
\usepackage{authblk}
\usepackage{url}
\usepackage[sorting=none]{biblatex}
\addbibresource{data-hazards.bib}


\title{
  Elizabeth Blackwell Institute Project Report:\\
  Stroke Imaging And Clinical Database For Artificial
  Intelligence
}
\author[1]{Emma Si\^{a}n Kuwertz}
\author[2,3]{Philip Clatworthy}
\affil[1]{Jean Golding Institute, University of Bristol}
\affil[2]{North Bristol NHS Trust}
\affil[3]{Bristol Medical School, University of Bristol}
\date{2021}

\begin{document}
\maketitle
%\tableofcontents
\section{Introduction}

Stroke is a major cause of disability in adults. Treatment
decisions rely on the ability of clinicians to combine insights
gleaned from medical imaging (e.g. computerized tomography (CT)) with
relevant clinical information and patient medical histories to anticipate the outcome of different treatment
scenarios. Decision support tools exist, but these are typically able
to consider information from isolated sources and do not allow the
integration of medical images. The \textit{Stroke Imaging and Clinical
Database for AI} project aims to extract provide a linked dataset,
using clinical and medical imaging inputs from North Bristol NHS Trust (NBT). 
This will facilitate the development of multi-modal models to explore the potential for stroke treatment outcome
prediction.\\



\section{Project achievements}



\subsection{Patient Public Involvement}

This project includes the use of patient information collected at NBT as a part
of routine care. Public patient involvement (PPI) is an essential
requirement of this research, and will help to ensure that the project
will be designed and implemented in such a way as to be acceptable to
the public. Community engagement also enhances the ethical
review process, facilitating open discussions about the risks and
benefits of the research and potentially flagging issues that can be
addressed early on in the project development.\\

As a part of internal ethics review and public consultation, this
project was presented and discussed at the first \textit{Data Hazards
  Workshop}~\cite{data-hazards-workshop} in September 2021. Here, this
project was presented and discussed within a small group of public
participants consisting of doctoral students, academics and data
professionals. An article
reporting on the discussions and considerations raised at the
workshop was subsequently produced and will form an important part of
the project PPI documentation.\\

The data hazards discussions highlighted concerns regarding the
following data hazards:

\begin{itemize}
\item Reinforcing existing bias
  \begin{itemize}
  \item The dataset is selected based on locality (limited to NBT)
    and procedure (neuroimaging) and so will be demographically
    biased.
  \item The sample may be intrinsically biased due to internal composition.
  \end{itemize}
\item Danger of misuse
  \begin{itemize}
  \item The immediate project output is a dataset and not an
    algorithm, therefore it is difficult to predict all possible
    future uses for this dataset. How can misuse be foreseen and prevented?
  \end{itemize}
\item Lack of informed consent
  \begin{itemize}
  \item Data used here is collected as a part of routine care, making
    the collection of explicit consent from individuals difficult / unrealistic.
  \end{itemize}     
\item Lack of community involvement
  \begin{itemize}
  \item In the absence of explicit patient consent, consultation with
    patient communities is important and should be a priority.
  \end{itemize}     
\end{itemize}


\section{Next steps}


\subsection{Patient Public Involvement}

\subsection{Consultation workshops}

Once the project data is in hand and in-principle available on the
University of Bristol's RDSF, a series of consultation workshops are
planned. The purpose of these will be as follows:

\begin{enumerate}
\item Gain feedback from the academic and clinical
  community on how they envisage using the dataset in their
  research. 
\item Identify additional research questions that will be studied using the data
  resource.
\item Understand how clinical and computer science researchers
  currently interact with data: which tools are predominantly used?
\item Gain insight into how researchers will want to query the
  data.
\item Identify any core data fields which will be most commonly
  in use.
\end{enumerate}

The above points will further enable the design and specification for
a queriable database. Addressing the above will allow planning for
optimal data extraction and filtering from the database, and will
essentially inform database design and construction.``

The initial preliminary workshop will be an opportunity to bring
together all co-applicants on this project and provide interactions
between clinical researchers, software engineers and computer
scientists. This will be an opportunity to present this preliminary
data offering and scope use cases and collaborative pursuits. 

\textbf{\underline{Preliminary workshop}}\\

\textbf{Foreseen attendees:}
\begin{itemize}
\item Dr Marcus Bradley, Consultant interventional neuroradiologist
  and Clinical Lead for Neuroradiology, North Bristol Trust
\item Rhona Galt, Director of Transformation and Innovation/AI lead,
  North Bristol Trust
\item Dr Emma Kuwertz, Data Science Specialist, Jean Golding
  Institute, University of Bristol
\item Dr Phil Clatworthy, Honorary Senior Lecturer and
  Consultant Stroke Neurologist, North Bristol Trust
\item Prof. Kate Robson-Brown, Director, Jean Golding
  Institute, University of Bristol
\item Mr Kumar Abhinav, Consultant Neurosurgeon, North Bristol Trust
\item Prof. Rob Hinchliffe, Professor of Vascular Surgery and
  Consultant Vascular Surgeon, North Bristol Trust
\item Dr Richard Ibitoye, Honorary Consultant Neurologist,
  Gloucestershire Hospitals NHS Foundation Trust
\item Prof Majid Mirmehdi, Professor of Computer Vision, Department of
  Computer Science, University of Bristol
\item Dr Edwin Simpson, Lecturer of Computer Science, Department of
  Computer Science, University of Bristol
\item Dr Conor Houghton, head of Neural Computation group, Faculty
  of Engineering, University of Bristol
\item Dr George Harston, Chief Medical and Innovation Officer,
  Brainomix Limited, Oxford, UK
\item \textit{Invite research software engineering representatives from
    University of Bristol's Advanced Computing Research Centre}
\end{itemize}

\textbf{Agenda}
\begin{itemize}
\item    Present data specification and availability \textit{(Phil
    Clatworthy \& Emma Kuwertz)}
\item    Introduce JGI and current health data science collaboration \textit{(Phil
    Clatworthy, Emma Kuwertz, Kate Robson-Brown)}
\item    Present research questions in stroke \textit{(Phil
    Clatworthy \& others)}, vascular surgery \textit{(Rob Hinchliffe
    \& others)},
  neuroradiology \textit{(Marcus Bradley \& others)}
\item Questions to participants (in advance):
  \begin{itemize}
  \item Participant research ideas
  \item Participant research area questions / problems / issues that
    could be addressed (or studies that could be facilitated) using
    the project data resource.
  \end{itemize}
\end{itemize}
    Plan for a series of workshops / hackathons to come up with and explore more concrete themes/projects.

The discussion surrounding the reinforcement of bias, given the
composition of the dataset, underlined the need for clinical expertise
during database construction. Domain knowledge should be exploited to
highlight potential selection bias in the data sample that might
otherwise be subtle. In order to better understand and document
population bias within the sample itself, it would be beneficial to
publish statistics relating to relative populations of the sensitive data
categories contributing to the sample. For example, the relative
proportion of male and female patients, and the relative proportions
of persons in given age ranges, would be useful to identify where the dataset might be biased towards a certain
sex or age demographic.
It would be important to communicate these biases clearly to data
users to ensure an awareness of the limitations of the dataset.\\
 
Although the discussion relating to potential misuse of the data
highlighted the impossibility of identifying every possible future
use, it did raise concerns about the availability of linked data.
The example of using brain-imaging to predict intelligence, and the
sort of damage that might do if applied in such a way as to automate
decision making or to classify individuals, was a powerful one. The
database envisaged for this work would not include information
pertaining to an individual's intelligence, but what information might
it include? This reinforces the idea that it is necessary to carefully
consider and define the data fields made available in the dataset that
is eventually made available to researchers. Consultation with
domain expertise will be vital to determine which fields are most
relevant to the clinical questions to be addressed by the dataset.\\

There was speculation as to what degree the general public expect that
their personal and medical information might be used in research
without their explicit permission. Reviewing ethical research
literature can offer insights in this space, where there have been
studies into how patients feel about what does and doesn't warrant
their explicit consent. In one case where dataset linkage is addressed
specifically~\cite{xafis}, it was noted that most people consulted
felt that data linkage projects without consent are acceptable,
with the condition that researchers do not gain access to identifiable
information. This underlines the importance of having a fully
anonymized dataset for researchers and for data linkage to take place
within the NBT system, and speaks to the necessity for public trust in data
stewards. In this project NBT are the data stewards of the sensitive
data, and the NHS is generally perceived as trustworthy organisation in terms
of information privacy and security.  \\

Even with available ethics literature and historical studies that seek
to understand public views on the processing of their data in a
health-care setting and beyond, the fruitful discussions highlighted
the need for engaging with both the public and care-givers early
throughout the project lifetime. The workshop served as a valuable
consultation with academics and data-centric professionals, and
identified key ethical points to consider and integrate into the
project design and implementation. 


%\printbibliography

\end{document}
