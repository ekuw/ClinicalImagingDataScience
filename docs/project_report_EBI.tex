\documentclass{article}

\usepackage[margin=1.0in]{geometry}
\usepackage{authblk}
\usepackage{url}
\usepackage[sorting=none]{biblatex}
\addbibresource{data-hazards.bib}


\title{
  Elizabeth Blackwell Institute Project Report:\\
  Stroke Imaging And Clinical Database For Artificial
  Intelligence
}
\author[1]{Emma Si\^{a}n Kuwertz}
\author[2,3]{Philip Clatworthy}
\affil[1]{Jean Golding Institute, University of Bristol}
\affil[2]{North Bristol NHS Trust}
\affil[3]{Bristol Medical School, University of Bristol}
\date{2021}

\begin{document}
\maketitle
%\tableofcontents
\section{Introduction}

Stroke is a major cause of disability in adults. Treatment
decisions rely on the ability of clinicians to combine insights
gleaned from medical imaging (e.g. computerized tomography (CT)) with
relevant clinical information and patient medical histories to anticipate the outcome of different treatment
scenarios. Decision support tools exist, but these are typically able
to consider information from isolated sources and do not allow the
integration of medical images. The \textit{Stroke Imaging and Clinical
Database for AI} project aims to extract provide a linked dataset,
using clinical and medical imaging inputs from North Bristol NHS Trust (NBT). 
This will facilitate the development of multi-modal models to explore the potential for stroke treatment outcome
prediction.\\



\section{Project achievements}

This project was granted funding for 6 months by the Elizabeth
Blackwell Institute (EBI) Health Data Science research strand (with
support from the Wellcome Trust Institutional Strategic Support Fund
(ISSF)), commencing in June 2021. During this 6 month period the
following aims were pursued:
\begin{enumerate}
\item Draft project protocol to gain NBT sponsorship.
\item Prepare documents for Health Research Authority approval.
\item Produce a specification for the data repository.
\item Conduct a small pilot project using data.
\item Scope opportunities for further funding.
\end{enumerate}

\subsection{Protocol and sponsorship}

The protocol for this project has been drafted and was submitted to
NBT research sponsor for review on 16/11/2021 (R\&I number 5055). This was prepared with
support from NBT. A risk register was completed

\subsection{Patient Public Involvement}

This project includes the use of patient information collected at NBT as a part
of routine care. Public patient involvement (PPI) is an essential
requirement of this research, and will help to ensure that the project
will be designed and implemented in such a way as to be acceptable to
the public. Community engagement also enhances the ethical
review process, facilitating open discussions about the risks and
benefits of the research and potentially flagging issues that can be
addressed early on in the project development.\\

As a part of internal ethics review and public consultation, this
project was presented and discussed at the first \textit{Data Hazards
  Workshop}~\cite{data-hazards-workshop} in September 2021. Here, this
project was presented and discussed within a small group of public
participants consisting of doctoral students, academics and data
professionals. An article
reporting on the discussions and considerations raised at the
workshop was subsequently produced and will form an important part of
the project PPI documentation.\\

The data hazards discussions highlighted concerns regarding the
following data hazards:

\begin{itemize}
\item Reinforcing existing bias
  \begin{itemize}
  \item The dataset is selected based on locality (limited to NBT)
    and procedure (neuroimaging) and so will be demographically
    biased.
  \item The sample may be intrinsically biased due to internal composition.
  \end{itemize}
\item Danger of misuse
  \begin{itemize}
  \item The immediate project output is a dataset and not an
    algorithm, therefore it is difficult to predict all possible
    future uses for this dataset. How can misuse be foreseen and prevented?
  \end{itemize}
\item Lack of informed consent
  \begin{itemize}
  \item Data used here is collected as a part of routine care, making
    the collection of explicit consent from individuals difficult / unrealistic.
  \end{itemize}     
\item Lack of community involvement
  \begin{itemize}
  \item In the absence of explicit patient consent, consultation with
    patient communities is important and should be a priority.
  \end{itemize}     
\end{itemize}


\section{Next steps}


\subsection{Patient Public Involvement}


During the expert consultation at the Data hazards workshop there was speculation as to what degree the general public expect that
their personal and medical information might be used in research
without their explicit permission (even where data is collected as a
part of routine care).
In general there was a notion that the public generally accepted that
their medical data would likely be ued to further the public good,
with the understanding that no identifiable information would be
compromised. Still, it is important to engage with patients in this
space. The Bristol Health Partners Stroke Health Integration Team
(HIT) has a Service User Group involving people across BNSSG with
experience of stroke and stroke services.
They provide PPI input for research projects and will be given the
opportunity to be involved in this project in the future,
particularly in the areas of research design, analysis and
dissemination.

\subsection{Consultation workshops}

Once the project data is in hand and in-principle available on the
University of Bristol's RDSF, a series of consultation workshops are
planned. It is envisaged that this series will be composed of one
preliminary workshop, followed by two hackathon-style events.
The purpose of these will be as follows:

\begin{enumerate}
\item Gain feedback from the academic and clinical
  community on how they envisage using the dataset in their
  research. 
\item Identify additional research questions that will be studied using the data
  resource.
\item Understand how clinical and computer science researchers
  currently interact with data: which tools are predominantly used?
\item Gain insight into how researchers will want to query the
  data.
\item Identify any core data fields which will be most commonly
  in use.
\end{enumerate}

The above points will further enable the design and specification for
a queriable database. Addressing the above will allow planning for
optimal data extraction and filtering from the database, and will
essentially inform database design and construction.\\

The initial preliminary workshop will be an opportunity to bring
together all co-applicants on this project and provide interactions
between clinical researchers, software engineers and computer
scientists. This will be an opportunity to present this preliminary
data offering and scope use cases and collaborative pursuits. The plan
for the preliminary workshop is as follows:\\

\noindent\textbf{\underline{Preliminary workshop}}\\

\textbf{Foreseen attendees:}
\begin{itemize}
\item Dr Marcus Bradley, Consultant interventional neuroradiologist
  and Clinical Lead for Neuroradiology, North Bristol Trust
\item Rhona Galt, Director of Transformation and Innovation/AI lead,
  North Bristol Trust
\item Dr Emma Kuwertz, Data Science Specialist, Jean Golding
  Institute, University of Bristol
\item Dr Phil Clatworthy, Honorary Senior Lecturer and
  Consultant Stroke Neurologist, North Bristol Trust
\item Prof. Kate Robson-Brown, Director, Jean Golding
  Institute, University of Bristol
\item Mr Kumar Abhinav, Consultant Neurosurgeon, North Bristol Trust
\item Prof. Rob Hinchliffe, Professor of Vascular Surgery and
  Consultant Vascular Surgeon, North Bristol Trust
\item Dr Richard Ibitoye, Honorary Consultant Neurologist,
  Gloucestershire Hospitals NHS Foundation Trust
\item Prof Majid Mirmehdi, Professor of Computer Vision, Department of
  Computer Science, University of Bristol
\item Dr Edwin Simpson, Lecturer of Computer Science, Department of
  Computer Science, University of Bristol
\item Dr Conor Houghton, head of Neural Computation group, Faculty
  of Engineering, University of Bristol
\item Dr George Harston, Chief Medical and Innovation Officer,
  Brainomix Limited, Oxford, UK
\item \textit{Invite research software engineering representatives from
    University of Bristol's Advanced Computing Research Centre}
\end{itemize}

\textbf{Agenda}
\begin{itemize}
\item    Present data specification and availability \textit{(Phil
    Clatworthy \& Emma Kuwertz)}
\item    Introduce JGI and current health data science collaboration \textit{(Phil
    Clatworthy, Emma Kuwertz, Kate Robson-Brown)}
\item    Present research questions in stroke \textit{(Phil
    Clatworthy \& others)}, vascular surgery \textit{(Rob Hinchliffe
    \& others)},
  neuroradiology \textit{(Marcus Bradley \& others)}
\item Questions to participants (in advance):
  \begin{itemize}
  \item Participant research ideas
  \item Participant research area questions / problems / issues that
    could be addressed (or studies that could be facilitated) using
    the project data resource.
  \end{itemize}
\end{itemize}

It is expected that discussions during the preliminary workshop will
result in the identification of additional data preparation /
processing steps necessary to facilitate researcher interaction with
the dataset. Ahead of the hackathons, the following steps will be
taken and resources
prepared for participants:

\begin{itemize}
\item Data quality checks and cleaning will be performed
\item Data will be linked together appropriately and access to the
  dataset on RDSF will be secured for named researchers
\item Metadata will be prepared to describe the dataset contents and layout
\item Data querying examples/scripts will be provided for hackathon participants
\item Instructions for data access will be provided to hackathon participants
\end{itemize}

During the hackathons, researchers will access the data and begin
preliminary analyses / investigations in their chosen research
area. This hands-on experience with the dataset will help to expose
any additional issues with the dataset and identify any missing or
perhaps superfluous information. In this way, the first hackathon is
expected to further inform dataset design. \\

The second hackathon will occur following a first round of feedback
from researchers regarding their experience of using the dataset. This
means that in the second hackathon access to a new iteration of the
dataset will be provided for researchers. Researchers will also
provide feedback on important questions like sample-size and
composition appropriateness. 

\subsection{Privacy and ethics}

There are several areas under ethics, privacy and security that will
be further studied and addressed. These include data investigations
and subsequent metadata to document data composition and highlight
internal dataset bias and arrangements for future data sharing and data access.

\subsubsection{Sample bias}
The discussions from the Data Hazards workshop highlighted the need
for clear documentation of sample composition and available data
fields. An important question to be investigated with the
researcher-access database is \textit{``what metadata should be provided
  alongside the database to facilitate the mitigation of bias when the
  data is used in research?"} At the very least the following
database statistics will be made available:
\begin{itemize}
\item What is the sample size?
\item What is the gender composition?
\item What is the ethnicity composition?
\item What is the hospital-level composition?
\item What is the procedural composition?
\item What is the age composition?
\end{itemize}

\noindent Clinicians will beconsulted to exploit domain knowledge that may highlight clinical
selection bias.


\subsubsection{Misuse}

To guard against potential misuse of this data resource, a clear data
access request and approval procedure will be drafted. It is envisaged
that new access requests will be submitted to the project PI (Phil
Clatworthy). These requests will need to include a clear proposal
describing the intended use of the data in research. If granted, access to the
data will be permitted for purposes related to the proposed research
only, and will be given for the proposed project duration.\\

To minimise the potential for data to be linked with other unintended data
sources out of context, the data fields made available in the dataset
will be strictly controlled. The planned consultation workshops will
provide a pruning exercise, whereby superfluous data fields will be
dropped and motivated data fields kept. \\

\subsection{Security}

Further engagement with NBT and University of Bristol IT services will
help to inform and solidify the data sharing arrangements for the
research database. The RDSF will be used at the University of Bristol
during dataset scoping and pilot phases, however a queriable database
would be more efficiently accessed from elsewhere (RDSF is not
optimised for efficient data access). Planning and executing this
phase of database design will be important following initial dataset
preparation (e.g. in parallel with / informed by the series of
consultation workshops).


%\printbibliography

\end{document}
